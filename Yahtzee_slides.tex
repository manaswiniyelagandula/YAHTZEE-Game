\documentclass[14pt]{beamer}
\title{BVRIT HYDERABAD College of Engineering for Women \\ {\color{magenta}{ YAHTZEE}}}
\subtitle{\color{magenta}{Team 19}}
\date{\ October 20,2021}
\author[Bvrith]{ Y MANASWINI : 20WH1A6657 : AIML \\ H SAHITHYA : 20WH1A0495 : ECE \\ R RITHIKA : 20WH1A0449 : ECE \\ S SATHVIKA : 20WH1A0523 : CSE \\ G RAVI SRI CHANDANA : 20WH1A1215 : IT}
\usefonttheme{serif}
\usepackage{bookman}
\usepackage{hyperref}
\usepackage[T1]{fontenc}
\usepackage{graphicx}
\usecolortheme{orchid}
\beamertemplateballitem

\begin{document}
    \begin{frame}
        \titlepage
    \end{frame}
    \begin{frame}
	\frametitle{Introduction}
        \begin{itemize}
	    \item YAHTZEE is a game based on Poker.
	    \item The object of the game is to roll certain combinations  of numbers with five dice. 
	\end{itemize}
    \end{frame}

\begin{frame}
	\frametitle{Scoring categories}
	\begin{enumerate}
	    \item Six scoring categories in upper(1-6).
	    \item Seven scoring categories in lower(seven boxes).  
	\end{enumerate}
	\begin{itemize}
		\item 3 of a kind, 4 of a kind: score total of all dice.
		\item Full house(3 of a kind + pair): 25 points.
	\end{itemize}
\end{frame}

\begin{frame}
	\begin{itemize}
	   \item Small straight(4 in sequence): 30 points.
	    \item Large straight(5 in sequence): 40 points
	    \item Yahtzee(5 of a kind): 50 points
	    \item Chance(any dice): score total of all dice
	\end{itemize}
    \end{frame}
 \begin{frame}
	\frametitle{Score card}
\begin{center}
\includegraphics[scale=0.50]{scorecard.png}
\end{center}
\end{frame}

\begin{frame}
	\frametitle{Approach}
	\begin{itemize}
	    \item Roll up to three times each turn to rack up the best possible score.
	    \item Decide which dice combo you are going for.
	    \item After each turn write your score in one empty box on the scorecard.
	\end{itemize}
\end{frame}

 \begin{frame}
        \frametitle{Learnings}
	\begin{itemize}
	    \item Learnt how to make presentations with latex.
	   \item Working with Git
	\end{itemize}
    \end{frame}
	
  \begin{frame}
	\frametitle{GIT Repo}
\begin{center}
\includegraphics[scale=0.17]{gitrepo.png}
\end{center}
\end{frame}
 \begin{frame}
	\frametitle{Statistics}
        \begin{itemize}
	     \item Number of Lines of Code: 166
	     \item Number of Functions: 5
        \end{itemize}
\end{frame}

 \begin{frame}
	\frametitle{Demo/ Screen Shots of the Project}
	\begin{center}
		\includegraphics[scale=0.07]{sc1.png}
		\includegraphics[scale=0.07]{sc2.png}
		\includegraphics[scale=0.07]{sc3.png}
		\includegraphics[scale=0.07]{sc4.png}
	\end{center}	
\end{frame}

\begin{frame}
	\begin{center}
		\includegraphics[scale=0.08]{sc5.png}
		\includegraphics[scale=0.08]{sc6.png}
	\end{center}
\end{frame}

 \begin{frame}
	\begin{center}
	    THANK YOU
	\end{center}
    \end{frame}
\enddocument
